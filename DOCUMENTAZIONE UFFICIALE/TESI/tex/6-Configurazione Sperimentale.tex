\section{Configurazione sperimentale}

\subsection{Metodologia di valutazione}

Valutiamo il framework LLM-Mob utilizzando un disegno sperimentale completo che cattura diversi aspetti delle prestazioni di previsione della mobilità:

\subsubsection{Costruzione del set di test}
Per ogni turista con almeno tre visite a POI, costruiamo istanze di previsione tramite:
\begin{enumerate}
\item Considerando la sequenza completa di visite $S = \{p_1, p_2, ..., p_n\}$
\item Utilizzando $\{p_1, p_2, ..., p_{n-1}\}$ come contesto di input
\item Prevedendo $p_n$ come target
\end{enumerate}

\subsubsection{Selezione del punto di ancoraggio}
Implementiamo strategie flessibili di selezione del punto di ancoraggio per studiare l'impatto di diversi punti di riferimento:
\begin{itemize}
\item \textbf{Penultimo}: Utilizza il penultimo POI visitato come posizione corrente
\item \textbf{Primo}: Utilizza il primo POI visitato come posizione corrente
\item \textbf{Medio}: Utilizza il POI centrale nella sequenza come posizione corrente
\item \textbf{Indice personalizzato}: Consente di specificare punti di ancoraggio arbitrari
\end{itemize}

\subsubsection{Metriche di prestazione}
Valutiamo le prestazioni di previsione utilizzando l'accuratezza Hit@K, dove una previsione è considerata corretta se il POI di base compare nelle prime K posizioni previste. Il nostro obiettivo principale è l'accuratezza Hit@5 per valutare l'utilità pratica del sistema.

\subsection{Configurazioni sperimentali}

I nostri esperimenti sono progettati per indagare i seguenti quesiti di ricerca:

\begin{enumerate}
\item In che modo l'inclusione di informazioni geografiche influisce sull'accuratezza della previsione?
\item Qual è l'impatto del contesto temporale sulle prestazioni di previsione della mobilità?
\item In che modo diversi cluster comportamentali turistici influenzano i risultati della previsione? \item Quale ruolo gioca la prossimità geografica nei modelli di spostamento turistico?
\end{enumerate}

