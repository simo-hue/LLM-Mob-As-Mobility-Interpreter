\chapter{Introduzione}

La previsione della mobilità umana si è affermata come un'area di ricerca cruciale, 
con applicazioni che spaziano dalla pianificazione urbana ai sistemi di trasporto, 
dai sistemi di raccomandazione alla salute pubblica. 
Gli approcci tradizionali alla previsione della mobilità si sono basati in larga misura 
su modelli statistici, algoritmi di apprendimento automatico e architetture di deep learning 
specificamente progettate per dati sequenziali. 
Tuttavia, i recenti progressi nei Large Language Model (LLM) hanno aperto nuove strade 
per comprendere e prevedere i modelli di comportamento umano attraverso 
capacità di elaborazione del linguaggio naturale.

La sfida fondamentale nella previsione della mobilità umana risiede 
nel catturare la complessa interazione tra fattori spaziali, temporali e comportamentali 
che influenzano le decisioni di movimento individuali. 
Gli approcci tradizionali di apprendimento automatico richiedono spesso 
un'ampia progettazione di feature e conoscenze specifiche del dominio 
per codificare efficacemente queste relazioni. 
Al contrario, i LLM possiedono capacità intrinseche di comprensione 
delle relazioni e dei modelli contestuali, 
che possono essere sfruttate per prevedere i modelli di mobilità 
attraverso informazioni contestuali e prompt attentamente progettati.

Questo lavoro presenta un framework che sfrutta la potenza 
dei Large Language Model per la previsione della mobilità umana in contesti turistici. 
L’approccio si differenzia dai metodi convenzionali perché tratta la previsione 
della mobilità come un compito di comprensione del linguaggio naturale, 
in cui i nomi dei POI storici, le informazioni geografiche e il contesto temporale 
vengono codificati come prompt testuali per l'inferenza LLM.

I principali contributi di questo lavoro includono:
\begin{itemize}
    \item Un nuovo framework per la previsione della mobilità umana utilizzando modelli di linguaggio (LLM);
    \item Valutazione completa di diversi tipi di informazioni contestuali sull'accuratezza della previsione;
    \item Implementazione utilizzando modelli open source, eliminando le dipendenze da API proprietarie e a pagamento;
    \item Esperimenti approfonditi su dati turistici reali tratti dal dataset VeronaCard;
    \item Analisi dell'impatto della prossimità geografica e dei modelli temporali sulle prestazioni di previsione della mobilità.
\end{itemize}