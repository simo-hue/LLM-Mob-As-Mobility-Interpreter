\chapter*{Abstract}
\addcontentsline{toc}{chapter}{Abstract}
Questo articolo presenta uno studio completo sull'utilizzo dei Large Language Models (LLM) per prevedere i modelli di mobilità umana nei contesti turistici urbani. Introduciamo LLM-Mob, un nuovo framework che sfrutta le capacità di comprensione del linguaggio naturale dei moderni LLM per prevedere i prossimi Punti di Interesse (POI) dei turisti in base ai loro modelli di visita storici. Il nostro approccio viene valutato sul dataset VeronaCard, che contiene dati turistici reali di Verona, Italia, relativi a diversi anni ( dal 2014 al 2023 ). Viene analizzato l'impatto di diversi tipi di informazioni contestuali sull'accuratezza della previsione, dai nomi di POI come base, coordinate geografiche e caratteristiche temporali. Il framework dimostra il potenziale degli LLM generalisti per comprendere complessi modelli spazio-temporali nella mobilità umana senza richiedere un'ingegnerizzazione approfondita delle caratteristiche o architetture specifiche per dominio. L'implementazione è partita sostituendo il framework di LLM-Mob che srfuttava i modelli OpenAI tramite API-KEY con invece modelli open source ( come Llama 3.1 ), rendendo l'approccio più accessibile ed economico per scopi di ricerca.