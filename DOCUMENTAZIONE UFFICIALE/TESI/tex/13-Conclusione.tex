\section{Conclusione}

Questo articolo presenta la potenza dei Large Language Model (LLM) generali (non ottimizzati) per la previsione della mobilità umana nella città di Verona.
Utilizzando il dataset VeronaCard, dimostriamo il potenziale degli LLM per comprendere complessi modelli spaziotemporali del comportamento turistico senza richiedere un'approfondita progettazione delle feature.

L'analisi di diversi tipi di informazioni contestuali rivela l'importanza del prompt, e in particolare del contesto geografico e temporale, nella previsione della mobilità. L'utilizzo di modelli open source (come Llama) da parte del framework lo rende accessibile a ricercatori e professionisti, mantenendo al contempo prestazioni competitive.

I nostri risultati contribuiscono al crescente corpus di lavori sulle applicazioni LLM nell'analisi spaziotemporale e forniscono le basi per la ricerca futura sui sistemi turistici intelligenti e sulla previsione della mobilità urbana.
