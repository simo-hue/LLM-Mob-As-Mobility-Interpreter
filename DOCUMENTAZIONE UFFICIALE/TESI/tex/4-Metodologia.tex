\section{Metodologia}

\subsection{Formulazione Del Problema}

Formuliamo il problema di previsione della mobilità umana come segue: data la sequenza storica di POI visitati da un turista $S = \{p_1, p_2, ..., p_n\}$ ordinati in base all'orario di visita, puntiamo a prevedere il prossimo POI $p_{n+1}$ che il turista probabilmente visiterà. Ogni POI $p_i$ è caratterizzato dal suo nome, dalle coordinate geografiche (latitudine e longitudine) e dalle informazioni temporali (timestamp della visita).

Il compito di previsione viene affrontato come un problema di generazione del linguaggio naturale, in cui al LLM vengono fornite informazioni contestuali sul comportamento del turista e viene chiesto di generare le prossime destinazioni più probabili.


\subsection{Descrizione Del Dataset}

I nostri esperimenti sono condotti sul dataset VeronaCard, che contiene dati turistici reali di Verona, Italia. Il dataset comprende i registri dei visitatori dal 2014 al 2023, registrando le interazioni dei turisti con vari siti culturali e storici della città. Il dataset include:
\begin{itemize}
\item Registri di utilizzo della tessera turistica con timestamp e identificativi dei POI
\item Un catalogo completo di Punti di Interesse con coordinate geografiche (a Verona)
\end{itemize}

Il catalogo dei POI include importanti attrazioni turistiche come l'Arena di Verona, la Casa di Giulietta e il Museo di Castelvecchio, tra le altre. Ogni POI è associato a precise coordinate geografiche, consentendo l'analisi spaziale dei modelli di movimento dei turisti.

\subsection{Pre-elaborazione e clustering dei dati}

La nostra pipeline di pre-elaborazione si compone di diverse fasi progettate per pulire e strutturare i dati grezzi sulla mobilità:

\subsubsection{Pulizia e filtraggio dei dati}
I log grezzi dei visitatori vengono elaborati per estrarre informazioni rilevanti, tra cui timestamp delle visite, nomi dei punti di interesse e identificativi delle tessere turistiche. Filtriamo i record incompleti e ci concentriamo sui turisti con più visite ai punti di interesse per garantire un'analisi sequenziale significativa (almeno 3 visite).

\subsubsection{Clustering degli utenti}
Per catturare i modelli comportamentali tra i turisti, utilizziamo il clustering K-means sulle matrici di interazione utente-punto di interesse. Il processo di clustering prevede:

\begin{enumerate}
\item Costruzione di una matrice utente-punto di interesse in cui ogni riga rappresenta un turista e ogni colonna rappresenta un punto di interesse
\item Standardizzazione della matrice utilizzando StandardScaler per garantire la stessa ponderazione delle caratteristiche
\item Applicazione del clustering K-means con $k=7$ cluster per raggruppare i turisti con modelli di visita simili
\end{enumerate}

Questo approccio di clustering consente l'identificazione di profili turistici distinti, come visitatori interessati alla cultura, appassionati di storia o turisti occasionali, che possono influenzare il processo di previsione.

