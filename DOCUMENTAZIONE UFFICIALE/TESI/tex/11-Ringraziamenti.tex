\section{Ringraziamenti}

Vorrei iniziare esprimendo la mia più profonda gratitudine alla mia relatrice di tesi, Sara Migliorini, il cui supporto, la cui guida esperta e il cui feedback costruttivo sono stati fondamentali per tutta la durata di questa ricerca/studio. Sono inoltre grato all'Università di Verona per avermi fornito l'accesso al prezioso dataset VeronaCard, che è servito da base per la validazione sperimentale presentata in questo studio.

Un ringraziamento speciale va a CINECA per avermi concesso l'accesso al supercomputer Leonardo attraverso l'iniziativa ISCRA (Italian SuperComputing Resource Allocation). Ringrazio CINECA per il premio ISCRA, per la disponibilità di risorse di calcolo ad alte prestazioni e per il supporto fornito. Gli esperimenti computazionali condotti in questa ricerca non sarebbero stati (così facilmente) possibili senza le eccezionali prestazioni della partizione Leonardo Booster. L'ambiente di calcolo ad alte prestazioni offerto da CINECA è stato essenziale per consentire la validazione sperimentale completa e il raggiungimento dei risultati presentati in questa tesi.

Sono particolarmente grato per l'opportunità di utilizzare un'infrastruttura computazionale così all'avanguardia, che non solo ha accelerato il processo di ricerca, ma mi ha anche permesso di esplorare progetti sperimentali più sofisticati e di condurre analisi più approfondite di quanto sarebbe stato possibile con risorse di calcolo convenzionali. Il supporto tecnico e la documentazione forniti dal team CINECA sono stati preziosi per affrontare le complessità degli ambienti di calcolo ad alte prestazioni.

Infine, esprimo la mia gratitudine a tutti i colleghi, amici e familiari che mi hanno supportato durante questo percorso accademico, offrendomi incoraggiamento e comprensione durante le fasi più impegnative di questa ricerca. Il loro supporto morale è stato una componente essenziale per la mia capacità di portare a termine con successo questo lavoro.
