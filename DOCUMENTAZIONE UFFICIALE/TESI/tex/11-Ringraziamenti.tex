\section{Ringraziamenti}

Desidero esprimere la mia più profonda gratitudine alla Prof.ssa Sara Migliorini, relatrice di questa tesi, per la sua guida esperta, il supporto costante e i preziosi consigli metodologici che hanno orientato questa ricerca dalle fasi iniziali fino alla stesura finale. La sua esperienza nel campo della mobilità urbana e dei sistemi intelligenti è stata fondamentale per definire l'approccio scientifico e interpretare criticamente i risultati ottenuti.

Ringrazio sentitamente l'Università di Verona per aver concesso l'accesso al dataset VeronaCard, risorsa di inestimabile valore che ha reso possibile la validazione sperimentale di questo studio. La disponibilità di dati reali di mobilità turistica ha rappresentato un elemento cruciale per testare l'efficacia del framework proposto in un contesto applicativo concreto.

Riconosco l'importanza CINECA nell'ambito dell'iniziativa ISCRA, per la disponibilità di risorse di calcolo ad alte prestazioni e relativo supporto che mi hanno reso autonomo ed indipendente per fare test senza dipendere dal server universitario. Gli esperimenti computazionali intensivi richiesti per l'addestramento e la valutazione dei Large Language Models sono stati condotti sulla partizione Leonardo Booster del supercomputer Leonardo. Le eccezionali capacità computazionali di questa infrastruttura hanno permesso l'esplorazione di architetture modellistiche complesse e la conduzione di analisi sperimentali su larga scala che non sarebbero state realizzabili con risorse convenzionali. Il supporto tecnico fornito dal team CINECA e la qualità della documentazione hanno facilitato significativamente l'utilizzo efficiente dell'ambiente HPC.

Un riconoscimento particolare va alla comunità scientifica internazionale che studia l'applicazione dei Large Language Models ai sistemi di raccomandazione e alla mobilità urbana. I contributi metodologici e i dataset pubblici messi a disposizione dai ricercatori in questo campo hanno fornito solide basi teoriche e strumenti di confronto per questo lavoro.

Esprimo inoltre la mia gratitudine ai colleghi del dipartimento e ai compagni di dottorato per i costruttivi scambi di idee, le discussioni metodologiche e il reciproco supporto durante lo sviluppo di questa ricerca. Le sessioni di confronto e peer review informale hanno contribuito significativamente al miglioramento della qualità del lavoro.

Infine, un ringraziamento sentito alla mia famiglia e agli amici più cari per la comprensione, l'incoraggiamento e il sostegno morale durante i momenti più impegnativi di questo percorso di ricerca. Il loro supporto incondizionato è stato una fonte di motivazione costante e ha reso possibile il completamento di questo lavoro con serenità e determinazione.