\section{Risultati e analisi}

\subsection{Confronto delle prestazioni tra i diversi tipi di contesto}

% Placeholder for results table
\begin{table}[H]
\centering
\caption{Prediction accuracy comparison across different contextual information types}
\label{tab:context_comparison}
\begin{tabular}{@{}lccc@{}}
\toprule
Context Type & Hit@1 & Hit@3 & Hit@5 \\
\midrule
POI Names Only & - & - & - \\
POI Names + Geography & - & - & - \\
POI Names + Geography + Temporal & - & - & - \\
\bottomrule
\end{tabular}
\end{table}

\subsection{Impatto delle informazioni geografiche}

L'integrazione delle informazioni geografiche rappresenta un significativo progresso nel nostro framework di previsione della mobilità. La nostra analisi rivela diverse intuizioni chiave:

\subsubsection{Previsioni basate sulla distanza}
L'inclusione di coordinate geografiche e calcoli delle distanze consente al LLM di effettuare previsioni più accurate basate sulla prossimità spaziale. I turisti in genere mostrano una preferenza per le attrazioni vicine e il nostro contesto geografico consente al modello di catturare efficacemente questi modelli.

\subsubsection{Analisi del clustering spaziale}
Osserviamo distinti modelli di clustering spaziale nel comportamento dei turisti, con alcune combinazioni di POI che mostrano tassi di co-occorrenza più elevati. Il contesto geografico aiuta il LLM a identificare questi modelli e a formulare previsioni in linea con i tipici comportamenti turistici.

% Placeholder for geographical analysis figure
\begin{figure}[H]
\centering
% \includegraphics[width=0.8\textwidth]{geographical_analysis.png}
\caption{Geographical distribution of POI predictions and their accuracy rates}
\label{fig:geographical_analysis}
\end{figure}

\subsection{Analisi dei pattern temporali}

L'integrazione di informazioni temporali fornisce un contesto aggiuntivo che migliora l'accuratezza delle previsioni:

\subsection{Effetti dell'ora del giorno}
La nostra analisi rivela che i pattern di visita variano significativamente in base all'ora del giorno, con alcuni POI più popolari in periodi specifici. Il contesto temporale consente all'LLM di catturare questi pattern sfumati.

\subsection{Variazioni stagionali}
Il set di dati pluriennale ci consente di analizzare gli effetti stagionali sul comportamento dei turisti, rivelando le preferenze per determinate attrazioni durante i diversi periodi dell'anno.

% Placeholder for temporal analysis figure
\begin{figure}[H]
\centering
% \includegraphics[width=0.8\textwidth]{temporal_analysis.png}
\caption{Temporal patterns in tourist mobility and their impact on prediction accuracy}
\label{fig:temporal_analysis}
\end{figure}

\subsection{Analisi del comportamento basata sui cluster}

L'approccio di clustering K-means rivela profili turistici distinti che influenzano i modelli di mobilità:

\subsection{Turisti culturali}
Un cluster mostra forti preferenze per i siti culturali e storici, con modelli di spostamento prevedibili tra le attrazioni correlate.

\subsection{Visitatori occasionali}
Un altro cluster mostra modelli di visita più diversificati, suggerendo turisti occasionali con interessi vari.

\subsection{Esploratori sistematici}
Un terzo cluster dimostra modelli di esplorazione sistematica, visitando attrazioni in prossimità geografica.

% Placeholder for cluster analysis figure
\begin{figure}[H]
\centering
% \includegraphics[width=0.8\textwidth]{cluster_analysis.png}
\caption{Tourist cluster characteristics and their impact on mobility prediction}
\label{fig:cluster_analysis}
\end{figure}
