\section{Discussione}

\subsection{Vantaggi dell'approccio basato su LLM}

Il framework LLM-Mob offre diversi vantaggi rispetto ai tradizionali approcci di previsione della mobilità:

\subsection{Ingegneria delle feature ridotta}
A differenza dei tradizionali approcci di apprendimento automatico che richiedono un'ingegnerizzazione delle feature estesa, il nostro metodo basato su LLM sfrutta le descrizioni in linguaggio naturale per codificare complesse relazioni spazio-temporali.

\subsection{Comprensione contestuale}
I LLM dimostrano capacità intrinseche di comprendere le relazioni contestuali tra le posizioni, consentendo previsioni più sfumate che considerano fattori che vanno oltre la semplice prossimità spaziale.

\subsection{Interpretabilità}
L'output in linguaggio naturale dei LLM fornisce spiegazioni interpretabili per le previsioni, offrendo spunti sul ragionamento alla base delle scelte di mobilità.

\subsection{Limitazioni e sfide}

Nonostante i suoi risultati promettenti, il framework LLM-Mob presenta diverse limitazioni:

\subsection{Requisiti computazionali}
L'esecuzione locale dei LLM richiede risorse computazionali significative, in particolare per l'inferenza accelerata da GPU.

\subsubsection{Sensibilità dei prompt}
Le prestazioni delle previsioni basate su LLM possono essere sensibili alla progettazione e alla formattazione dei prompt, richiedendo un'attenta progettazione dei contesti di input.

\subsubsection{Problemi di scalabilità}
L'elaborazione di grandi set di dati con gli LLM può richiedere molto tempo, soprattutto se confrontata con gli approcci tradizionali di apprendimento automatico.

