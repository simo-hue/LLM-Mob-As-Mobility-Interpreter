\chapter{Lavori correlati}

La previsione della mobilità umana è stata ampiamente studiata in diverse discipline, 
con approcci che spaziano dai modelli statistici alle architetture di deep learning. 
In questa sezione vengono riassunti i principali filoni di ricerca.

\section{Modelli statistici tradizionali}
I primi lavori si concentravano su modelli basati su catene di Markov, 
che catturavano le probabilità di transizione tra posizioni 
sulla base di dati storici di mobilità.

\section{Approcci di deep learning}
Con l’avvento del deep learning, reti neurali ricorrenti (RNN) e 
reti a memoria a lungo termine (LSTM) sono state utilizzate 
per modellare le dipendenze sequenziali nei dati. 
Successivamente, i meccanismi di attenzione hanno migliorato ulteriormente 
la capacità di rappresentare relazioni temporali complesse.

\section{Approcci basati su grafi}
Gli approcci basati su grafi hanno acquisito importanza grazie alla capacità 
di catturare relazioni spaziali tra posizioni. 
In questi metodi, le posizioni vengono rappresentate come nodi di un grafo 
e gli archi codificano prossimità spaziale o probabilità di transizione. 
Tali metodi, tuttavia, richiedono spesso pre-elaborazione complessa 
e ingegnerizzazione di feature specifiche del dominio.

\section{Uso dei Large Language Models}
L’emergere dei Large Language Model (LLM) ha introdotto nuove possibilità 
per la previsione della mobilità. 
Studi recenti hanno dimostrato la capacità degli LLM di comprendere 
relazioni spazio-temporali attraverso descrizioni in linguaggio naturale. 
Il presente lavoro si inserisce in questa linea, con particolare attenzione 
alla previsione della mobilità turistica e all’impatto 
di differenti tipologie di informazioni contestuali.