\section{Lavori correlati}

La previsione della mobilità umana è stata ampiamente studiata in diverse discipline, con approcci che spaziano dai modelli statistici alle architetture di deep learning. I primi lavori si concentravano su modelli basati su Markov che catturavano le probabilità di transizione tra posizioni sulla base di modelli storici. Approcci più recenti hanno sfruttato reti neurali ricorrenti (RNN), reti a memoria a lungo termine (LSTM) e meccanismi di attenzione per modellare le dipendenze sequenziali nei dati di mobilità.

Gli approcci basati su grafi hanno acquisito importanza grazie alla loro capacità di catturare le relazioni spaziali tra posizioni. Questi metodi in genere rappresentano le posizioni come nodi in un grafo, con archi che codificano la prossimità spaziale o le probabilità di transizione. Tuttavia, tali approcci richiedono spesso un'ampia pre-elaborazione e l'ingegnerizzazione di feature specifiche del dominio per ottenere prestazioni soddisfacenti.

L'emergere dei Large Language Model (LLM) ha introdotto nuove possibilità per la previsione della mobilità. Lavori recenti hanno esplorato l'uso dei LLM per vari compiti spazio-temporali, dimostrando la loro capacità di comprendere relazioni geografiche e modelli temporali attraverso descrizioni in linguaggio naturale. Il nostro lavoro si basa su queste basi, concentrandosi in particolare sulla previsione della mobilità turistica e analizzando l'impatto di diversi tipi di informazioni contestuali.
